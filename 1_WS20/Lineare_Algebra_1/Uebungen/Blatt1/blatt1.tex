% document setup
\documentclass[12pt,a4paper]{article}
\usepackage[utf8]{inputenc}
\usepackage[ngerman]{babel}

% maths
\usepackage{amsfonts}
\usepackage{amssymb}
\usepackage{amsmath}

% utility
\usepackage{xcolor}
\usepackage[colorlinks=false,linkbordercolor=red,urlbordercolor=red]{hyperref}
\usepackage[shortlabels]{enumitem}
\usepackage{tikz}

% useful commands
\newcommand{\qed}{\null\nobreak\hfill\square}

% title, author etc.
\title{Lineare Algebra I, Blatt 1}
\author{Gruppe 4\\
    Lorenz Bung (Matr.-Nr. 5113060)\\
    \href{mailto:lorenz.bung@students.uni-freiburg.de}{\texttt{lorenz.bung@students.uni-freiburg.de}}\\
    Tobias Remde (Matr.-Nr. 5100067)\\
    \href{mailto:tobias.remde@gmx.de}{\texttt{tobias.remde@gmx.de}}
}
\date{\today}

% begin document
\begin{document}

\maketitle


\section*{Aufgabe 1}

\begin{enumerate}[(a)]
    \item \[
    \begin{pmatrix}
        3 & 2 & -2\\
        2 & 1 & -3\\
        -2 & -1 & 2\\
    \end{pmatrix} \overset{S_{2,3}(1)}{\longrightarrow}
    \begin{pmatrix}
    3 & 2 & -2\\
    0 & 0 & -1\\
    -2 & -1 & 2\\
    \end{pmatrix} \overset{V_{2,3}}{\longrightarrow}
    \begin{pmatrix}
    3 & 2 & -2\\
    -2 & -1 & 2\\
    0 & 0 & -1\\
    \end{pmatrix}
    \]
    \[
    \overset{M_2(3)}{\longrightarrow}
    \begin{pmatrix}
    3 & 2 & -2\\
    -6 & -3 & 6\\
    0 & 0 & -1\\
    \end{pmatrix} \overset{S_{2,1}(2)}{\longrightarrow}
    \begin{pmatrix}
    3 & 2 & -2\\
    0 & 1 & 2\\
    0 & 0 & -1\\
    \end{pmatrix}
    \]
    Der Rang der Matrix ist die Zahl der nichttrivialen Zeilen. In diesem Fall sind dies $3$ Zeilen.\\
    Der Lösungsraum $\mathcal{S}$ lässt sich darstellen durch:
    \[
    \mathcal{S} = \left\{\left(\frac{b_1 - 2b_2 - 8b_3}{3}, b_2 + 2b_3, -b_3\right)\ \big|\ (b_1,b_2,b_3) \in \mathbb{R}^3\right\}
    \]

    \item \begin{equation}
    \label{a1:eq1}
    \begin{pmatrix}
    2 & 1 & -1 & 1\\
    -3 & -3 & 2 & 1\\
    -2 & 0 & 3 & -6\\
    5 & -3 & 2 & 1\\
    \end{pmatrix}
    \end{equation}
    Durch Hintereinderausführung der folgenden Funktionen ergibt sich aus Matrix (\ref{a1:eq1}) die Matrix (\ref{a1:eq2}):

    \[A_{3,1}(1) \rightarrow M_2(2) \rightarrow M_4(2) \rightarrow A_{2,1}(3) \rightarrow A_{4,1}(5) \rightarrow V_{2,3} \rightarrow A_{4,2}(1)\]
    \[\rightarrow A_{3,2}(3) \rightarrow V_{3,4} \rightarrow A_{4,3}(-7)\]

    \begin{equation}
    \label{a1:eq2}
    \begin{pmatrix}
    2 & 1 & -1 & 1\\
    0 & 1 & 2 & 5\\
    0 & 0 & 1 & 8\\
    0 & 0 & 0 & -40\\
    \end{pmatrix}
    \end{equation}

    Der Rang der Matrix ist $4$, da es $4$ nichttriviale Zeilen gibt.\\
    Der Lösungsraum $\mathcal{S}$ lässt sich darstellen durch:
    \[
    \mathcal{S} = \left\{\left(\frac{2b_1 - 2b_2 + 3b_3 + 79b_4}{4}, \frac{2b_2 - b_3 + 11b_4}{2}, b_3 + 5b_4, -40b_4\right)\ \big|\ (b_1, \dots, b_4) \in \mathbb{R}^4\right\}
    \]

\end{enumerate}


\section*{Aufgabe 2}

\begin{enumerate}[(a)]
    \item $(\mathbb{Q}, \star)$ ist Halbgruppe $\Leftrightarrow a \star (b \star c) = (a \star b) \star c \quad \forall a,b,c \in \mathbb{Q}.$\\
    Seien $a,b,c \in \mathbb{Q}$. Dann ist\\
    $a \star (b \star c) = a \star (cb + 2c + 2b + 2)$\\
    $= a(cb + 2c + 2b + 2) + 2a + 2(cb + 2c + 2b + 2) + 2$\\
    $= abc + 2ac + 2ab + 2bc + 4a + 4b + 4c + 6$\\
    $= (ab + 2a + 2b + 2)c + 2(ab + 2a + 2b + 2) + 2c + 2$\\
    $=(ab + 2a + 2b + 2) \star c$\\
    $= (a \star b) \star c.$\\
    Somit ist $(\mathbb{Q}, \star)$ eine Halbgruppe.\\\\
    Es bleibt zu zeigen, dass $(\mathbb{Q}, \star)$ abelsch ist:\\
    $(\mathbb{Q}, \star)$ abelsch $\Leftrightarrow a \star b = b \star a \quad \forall a,b \in \mathbb{Q}.$
    Seien also $a, b \in \mathbb{Q}$. Dann ist\\
    $a \star b = ab + 2a + 2b + 2$\\
    $= ba + 2b + 2a + 2 = b \star a.$\\
    $\qed$

    \item Aufgrund der Kommutativität ist ein linksneutrales Element auch das rechtsneutrale Element.
    Es genügt also zu zeigen, dass ein rechtsneutrales Element existiert.\\
    Seien also $a, b \in \mathbb{Q}$. Dann ist $b$ rechtsneutrales Element, wenn
    \[
    a \star b = ab + 2a + 2b + 2 = a.
    \]
    Durch Umformungen erhält man\\
    $ab + 2a + 2b + 2 = a$\\
    $\Leftrightarrow ab + a + 2b = -2$\\
    $\Leftrightarrow b (a + 2) + a = -2$\\
    $\Leftrightarrow b = - \frac{2 + a}{a + 2}$\\
    $\Leftrightarrow b = -1.$

    \item Seien $a, b \in \mathbb{Q}$. Dann ist $b$ (rechts-)inverses von $a$, falls
    \[
    a \star b = e = -1.
    \]
    Durch Umformungen erhält man\\
    $ab +2a +2b +2 = -1$\\
    $\Leftrightarrow b(a+2) +2a = -3$\\
    $\Leftrightarrow b(a+2) = -3 +2a$\\
    $\Leftrightarrow b = - \frac{3 + 2a}{a + 2}.$\\
    Für $a = 0$ erhält man somit $b = - \frac{3}{2}$.\\
    Für $a = -2$ hat die Gleichung keine Lösung, somit existiert auch kein inverses Element.
\end{enumerate}


\section*{Aufgabe 3}

\[
\begin{array}{c||c|c|c|c|}
\star & a & b & c & d\\
\hline\hline
a & b & a & d & c\\
b & a & b & c & d\\
c & d & c & b & a\\
d & c & d & a & b\\
\hline
\end{array}
\]

Das neutrale Element der Gruppe ist $b$.


% end document
\end{document}