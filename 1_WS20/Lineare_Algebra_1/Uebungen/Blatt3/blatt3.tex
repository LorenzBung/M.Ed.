% document setup
\documentclass[12pt,a4paper]{article}
\usepackage[utf8]{inputenc}
\usepackage[ngerman]{babel}

% maths
\usepackage{amsfonts}
\usepackage{amssymb}
\usepackage{amsmath}

% utility
\usepackage{xcolor}
\usepackage[colorlinks=false,linkbordercolor=red,urlbordercolor=red]{hyperref}
\usepackage[shortlabels]{enumitem}
\usepackage{tikz}

% useful commands and formatting
\newcommand{\qed}{\null\nobreak\hfill\square}
\setlength{\parindent}{0em}
\setlength{\parskip}{6pt}

% title, author etc.
\title{Lineare Algebra I, Blatt 3}
\author{Gruppe 4\\
    Lorenz Bung (Matr.-Nr. 5113060)\\
    \href{mailto:lorenz.bung@students.uni-freiburg.de}{\texttt{lorenz.bung@students.uni-freiburg.de}}\\
    Tobias Remde (Matr.-Nr. 5100067)\\
    \href{mailto:tobias.remde@gmx.de}{\texttt{tobias.remde@gmx.de}}
}
\date{\today}

% begin document
\begin{document}

\maketitle


\section*{Aufgabe 1}

\begin{enumerate}[(a)]
    \item Wir haben eine lineare Ordnung, d.h. entweder ist $0_R < a$ oder $a < 0_R$.

    \textbf{Fall 1}: $0_R < a$.\\
    Dann ist $0_R \cdot c < a \cdot c,\ c > 0_R$, da $<$ kompatibel mit $R$ ist.
    Wähle nun $c := a$, dann ist also $0_R \cdot a = 0_R < a \cdot a = a^2$ und somit $a^2$ positiv.

    \textbf{Fall 2}: $a < 0_R$.\\
    Dann ist $a + c < 0_R + c,\ c > 0_R$.
    Wir wählen $c := -a > 0_R$ und erhalten $a - a  = 0_R < 0_R -a = -a$.\\
    Damit erhalten wir $0_R \cdot (-a) = 0_R < (-a) \cdot (-a) = -((-a) \cdot a) = -(-(a^2)) = a^2$ und damit $a^2 = (-a)^2$ positiv.

    Zusammenfassend folgt, dass $a^2 > 0_R$ für alle $a \in R$.\\
    $\qed$

    \item \textbf{Behauptung}: Es gibt keine kompatible lineare Ordnung auf dem Körper $\mathbb{C}$.

    \textbf{Beweis}:
    Wir führen einen Widerspruchsbeweis.
    Angenommen, es gäbe eine solche kompatible lineare Ordnung.
    Dann haben wir in Teilaufgabe (a) schon gezeigt, dass Quadrate $\{a^2\}_{a \in \mathbb{C}}$ bezüglich dieser kompatiblen linearen Ordnung positiv sein müssen.\\
    Wir wählen nun $x := 0+1i \in \mathbb{C}$.
    Dann müsste $x^2$ positiv sein.
    Es ist aber $x^2 = i^2 = -1 < 0$.\\
    Widerspruch!
    Es folgt, dass es keine solche Ordnung geben kann.\\
    $\qed$

    \item \textbf{Behauptung}: Wenn $R$ positive Charakteristik hat, besitzt $R$ keine kompatible lineare Ordnung.

    \textbf{Beweis}:
    Wir führen einen Widerspruchsbeweis.
    Angenommen, es gäbe eine solche lineare Ordnung $\overset{\sim}{<}$.
    Dann wäre
    $$a \overset{\sim}{<} b \Rightarrow \left\{ \begin{array}{ll}a+c\overset{\sim}{<} b+c &\text{für alle }c\\ ac \overset{\sim}{<} bc &\text{falls } c \overset{\sim}{>} 0_R\end{array}\right..$$
    Da $\overset{\sim}{<}$ linear ist, ist entweder $0_R \overset{\sim}{<} 1_R$ oder $1_R \overset{\sim}{<} 0_R$.

    \textit{Fall 1}: $1_R \overset{\sim}{<} 0_R$.\\
    Wähle $c := 1_R + 1_R + \dots = \sum\limits_{i=0}^{k-1} 1_R$.
    Dann ist
    $$1_R + c = \sum\limits_{i=0}^{k} 1_R = 0_R \overset{\sim}{<} \sum\limits_{i=0}^{k-1} 1_R = 0_R + c.$$
    Widerspruch, da $1_R \overset{\sim}{<} 0_R$ war.

    \textit{Fall 2}: $0_R \overset{\sim}{<} 1_R$.\\
    Analog zu Fall 1.

    Damit kann keine solche Ordnung existieren.\\
    $\qed$
\end{enumerate}


\section*{Aufgabe 2}

\begin{enumerate}[(a)]
    \item Sei $a^2 = a$ für $a \in R$.
    Dann ist
    $$0_R = (0_R)^2 = (a-a)^2 = a^2 + 2a^2 + (-a)^2 = a + 2a -a = 2a$$
    was genau der Fall ist, wenn $0_R = a$.\\
    Wegen $a = aa = a^2$ ist auch $1_R = a$.\\
    Damit muss $\text{char}(R) = 1$ sein.\\
    $\qed$

    \item Seien $a, b \in R$.
    Dann ist nach Definition
    $$a+b = (a+b)^2 = a^2 + 2ab + b^2 = a + 2ab + b.$$
    Dies ist äquivalent zu $0 = ab$, was genau dann der Fall ist, wenn $a$ das Inverse (bezüglich der Multiplikation) von $b$ und $b$ das Inverse von $a$ ist, also $a = b^{-1}$ und $b = a^{-1}$.\\
    Dann ist jedoch $ab = aa^{-1} = 0_R = bb^{-1} = ba$ und damit die Multiplikation kommutativ.\\
    $\qed$
\end{enumerate}


\section*{Aufgabe 3}

Die Abbildung
$$f := (R, +) \rightarrow (R, +), x \mapsto x^p$$
mit $\text{char}(R) = p > 0$ ist genau dann ein Gruppenhomomorphismus, wenn
$$f(a + b) = f(a) + f(b),\ a,b \in R$$
gilt.

Es ist
$$f(a + b) = (a + b)^p = \sum_{k=0}^{p}\binom{p}{k}a^{p-k}b^k = a^p + \sum_{k=1}^{p-1}\binom{p}{k}a^{p-k}b^k + b^p.$$
Dies lässt sich auch schreiben als
$$a^p + \sum_{k=1}^{p-1}p \cdot \frac{(p-1)!}{k!(p-k)!} a^{p-k}b^k + b^p.$$
$\frac{(p-1)!}{k!(p-k)!} a^{p-k}b^k$ ist jedoch ein Element aus dem Ring, und da $p$ die Charakteristik des Ringes ist, werden alle dieser Summanden $0$ (folgt aus \textit{Definition 1.31.} im Skript).

Es bleibt $f(a+b) = (a+b)^p = a^p + b^p = f(a) + f(b)$.
Somit ist $f$ ein Gruppenhomomorphismus.\\
$\qed$

\pagebreak
\section*{Aufgabe 4}

\begin{enumerate}[(a)]
    \item Die Vektoren $v_1 = 3 + 4i$ und $v_2 = 1 - 2i$ sind genau dann linear abhängig, wenn $\lambda_1,\lambda_2 \in \mathbb{R}$ bzw. $\lambda_1,\lambda_2 \in \mathbb{C}$ existieren, sodass die Gleichung $\lambda_1v_1 + \lambda_2 v_2 = 0$ eine Lösung mit $\lambda_1,\lambda_2 \neq 0$ hat.\\
    Dies ist genau dann der Fall, wenn ein $\lambda \in \mathbb{R}$ bzw. $\lambda \in \mathbb{C}$ existiert mit $\lambda = \frac{\lambda_1}{\lambda_2}$.

    Die Multiplikation eines Vektors $a + bi \in \mathbb{C}$ mit einem Skalar $c \in \mathbb{R}$ resultiert in einem Vektor $ac +bci \in \mathbb{C}$. Bei Wahl eines Skalars $c + di \in \mathbb{C}$ entsteht jedoch der Vektor $ac - bd + (ad + bc)i \in \mathbb{C}$.

    Die resultierenden Gleichungssysteme sind zwar in $\mathbb{C}$ lösbar, jedoch nicht in $\mathbb{R}$. Somit sind die Vektoren $v_1$ und $v_2$ im $\mathbb{C}$-Vektorraum $\mathbb{C}$ linear abhängig, aber linear unabhängig im $\mathbb{R}$-Vektorraum $\mathbb{C}$.

    \item Analog zu Teilaufgabe (a) sind die beiden Vektoren $v_1 = 3 - 2\sqrt{2}$ und $1 + \sqrt{2}$ genau dann linear abhängig, wenn es ein $\lambda \in \mathbb{Q}$ bzw. $\lambda \in \mathbb{R}$ mit $\lambda \neq 0$ gibt, welches die Gleichung $\frac{v_1}{v_2} = \lambda$ löst.

    Im $\mathbb{R}$-Vektorraum $\mathbb{R}$ existiert ein solches $\lambda$, nämlich $7 + 5\sqrt{2} = \frac{1+\sqrt{2}}{3-2\sqrt{2}} = \frac{v_2}{v_1}$.
    In $\mathbb{Q}$ gibt es jedoch kein solches $\lambda$, da $7+5\sqrt{2} \notin \mathbb{Q}$.

    Somit sind die Vektoren $v_1$ und $v_2$ im $\mathbb{Q}$-Vektorraum $\mathbb{R}$ linear unabhängig, und im $\mathbb{R}$-Vektorraum $\mathbb{R}$ linear abhängig.
\end{enumerate}


% end document
\end{document}